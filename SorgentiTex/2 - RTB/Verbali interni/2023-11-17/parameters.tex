% Parametri che modificano il file main.tex
% Le uniche parti da cambiare su main.tex sono:
% - vari \vspace tra sezioni
% - tabella azioni da intraprendere
% - sezione altro

\def\data{2023-11-17}
\def\oraInizio{09:00}
\def\oraFine{10:00}
\def\luogo{Piattaforma Discord}

\def\tipoVerb{Interno} % Interno - Esterno

\def\nomeResp{Feltrin E.} % Cognome N.
\def\nomeVer{Campese M.} % Cognome N.
\def\nomeSegr{Ciriolo I.} % Cognome N.

\def\nomeAzienda{Azzurro Digitale}
\def\firmaAzienda{azzurrodigitale.png}
\def\firmaResp{emanuele.png} % nome Responsabile

\def\listaPartInt{
Bresolin G.,
Campese M.,
Ciriolo I.,
Dugo A.,
Feltrin E.,
Michelon R.,
Orlandi G.
}

\def\listaPartEst{
Azzurro B.,
Digitale C.,
}

% Se nessuna revisione: \def\listaRevisioneAzioni {x}
\def\listaRevisioneAzioni {
{Feedback sull'assegnazione ruoli (avvenuta in precedenza) da parte di tutti i membri del team;},
{Controllo dei documenti in produzione con particolare attenzione verso l'impostazione del Piano di qualifica, Piano di progetto ed Analisi dei requisiti;},
{Verifica dei verbali prodotti durante la settimana.}
}

\def\listaOrdineGiorno {
{Assegnazione ruoli riunione;},
{Aggiornamento sul lavoro individuale in corso da parte del team di amministratori (Bresolin G., Dugo A., Michelon R.) ed analisti (Ciriolo I., Orlandi G.);},
{Assegnazione task per la settimana successiva;},
{Breve spiegazione tenuta da Dugo A. relativa all'utilizzo del software \textit{Jira};},
{Presa in esame dell'idea proposta dall'azienda AzzurroDigitale di utilizzare di un \textit{repository} messo a disposizione da loro.}
}

\def\listaDiscussioneInterna {
{Assegnazione ruoli riunione:
    \begin{itemize}
        \item Responsabile: Feltrin E.;
        \item Segretario di riunione: Ciriolo I.;
        \item Responsabile della revisione: Campese M.
    \end{itemize}},
{Durante l'aggiornamento, il team di amministratori presenta alcuni cambiamenti effettuati nella gestione e versionamento della documentazione; segue la presentazione delle bozze dei diagrammi dei Casi d'uso da parte di Ciriolo I.;},
{A seguito di una breve discussione sono state assegnate le azioni da intraprendere che si trovano nell'apposita sezione a pagina seguente;},
{Successivamente alla spiegazione sull'utilizzo di \textit{Jira} si è pensato di utilizzarlo per l'organizzazione del lavoro del team. Questo software permetterebbe di specificare le dipendenze nei \textit{diagrammi di Gannt} e imporrebbe di chiudere tutti i task di una \textit{milestone} prima che questa possa essere considerata raggiunta;},
{Si è discussa la possibilità di utilizzare il \textit{repository} offerto dall'azienda proponente, notando però che non verrebbero stanziate particolari agevolazioni per il team, che dispone già di un \textit{repository} avviato.}
}

\newcommand{\domris}[2]{\textbf{#1}\\#2}

\def\listaDiscussioneEsterna {
\domris
{Domanda 1
}
{Riposta 1;
},
\domris
{Domanda 2
}
{Risposta 2;
},
\domris
{Domanda 3
}
{Risposta 3.
}
}

% Se nessuna decisione: \def\listaDecisioni {x}
\def\listaDecisioni {
{Si è deciso definitivamente di non utilizzare il repository messo a disposizione dall'azienda proponente;},
{Il team utilizzerà il software \textbf{\textit{Jira}} a partire da questa riunione in avanti. Lo spazio di lavoro sarà organizzato dagli amministratori Dugo A. e Bresolin G.;},
{Soppressione \textit{issue} per la creazione di templeta per le \textit{issue} stesse in seguito al passaggio su \textit{Jira};},
{I diagrammi dei Casi d'uso da inserire nell'Analisi dei requisiti saranno prodotti da Ciriolo I. e revisionati da Bresolin G.;},
{Il nome della repository utilizzata dal team sarà \textbf{Knowledge Managment AI}, nome del capitolato dell'azienda proponente;},
{Durante il weekend avrà luogo un workshop presentato dal team di amministratori sulla configurazione dell'ambiente di lavoro e sul funzionamento degli \textit{script} in \textit{Python}.}
}