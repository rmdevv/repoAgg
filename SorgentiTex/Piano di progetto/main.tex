\documentclass[10pt, a4paper]{article}

\input{parameters}

\usepackage{style}
\usepackage{headerfooter}
\usepackage{caption}
\title{\titolo}
\author{SWEetCode}

\begin{document}

% PRIMA PAGINA
\include{firstpage}

% REGISTRO DELLE VERSIONI
\include{registroversioni}
\newpage

% INDICE
\tableofcontents
% SOMMARIO
\section*{Sommario}
\listoffigures
\listoftables
\newpage

% INIZIO PAGINE

\section{Introduzione}


\subsection{Scopo del documento}
\paragraph{}Il documento ha l'obiettivo di definire le risorse da impiegare, le modalità e le tempistiche da seguire per lo svolgimento del progetto. In particolare viene effettuata un'\textbf{analisi dei rischi attesi}, a cui vengono affiancate delle \textbf{pratiche di mitigazione} degli stessi. Si propone inoltre una \textbf{valutazione dell'efficacia} di queste pratiche,così da portare ad eventuali miglioramenti o correzioni delle stesse nel caso in cui non dovessero portare ai risultati desiderati.
\paragraph{}Il documento si sviluppa poi nelle sezioni di \textbf{pianificazione} delle attività, indicando \textbf{preventivo} e \textbf{consuntivo} di ogni periodo e infine si conclude con la parte di \textbf{retrospettiva}, in cui vengono analizzate la gestione del tempo e del budget e le pratiche che si sono rivelate più o meno buone nel corso dello svolgimento del progetto.
\paragraph{}In aggiunta è necessario specificare che tale documento viene redatto con un approccio incrementale, in maniera tale da poter implementare facilmente dei cambiamenti nel corso del tempo a seconda delle necessità.



%\subsection{Scopo del prodotto}

\subsection{Glossario}
Per evitare ambiguità e incomprensioni relative al linguaggio e ai termini utilizzati nella documentazione relativa al progetto viene presentato un Glossario. I termini ambigui o specifici presenti nello stesso, verranno identificati con un pedice |g|. (DA VALUTARE COME RICONOSCERE IL TERMINE )
\subsection{Riferimenti}

 %SEZIONE RIF. NORMATIVI
\subsubsection{Riferimenti normativi} 
\begin{itemize}
\item \textit{Regolamento del progetto didattico}: \\
\href{https://www.math.unipd.it/~tullio/IS-1/2023/Dispense/PD2.pdf}{https://www.math.unipd.it/~tullio/IS-1/2023/Dispense/PD2.pdf}\\
(Ultimo accesso: $2023-11-13$);
\item \textit{Norme di progetto v0.4.3};
\item \textit{Analisi dei requisiti v1.0.1}.
\end{itemize}

% SEZIONE RIF. INFORMATIVI
\subsubsection{Riferimenti informativi}
\begin{itemize}
    \item \textit{Glossario v0.0.1} (da creare parallelamente) specificare la versione oppure come risorsa "web" per facilitare la consultazione rapida e agile quindi con data di ultimo accesso?
   
    \item \textit{Presentazione capitolato}:\\
    \href{https://www.math.unipd.it/~tullio/IS-1/2023/Progetto/C1.pdf}{https://www.math.unipd.it/~tullio/IS-1/2023/Progetto/C1.pdf}\\
    (Ultimo accesso: $2023-11-13$);   
    
    \item \textit{Verbali esterni ed interni};
    \item \textit{Preventivo costi e impegni orario v0.0.1(23)}
    
    \item \textit{Dispense su ciclo di vita del SW}:\\
    \href{https://www.math.unipd.it/~tullio/IS-1/2023/Dispense/T2.pdf}{https://www.math.unipd.it/~tullio/IS-1/2023/Dispense/T2.pdf}\\
    (Ultimo accesso: $2023-11-13$);
    
    \item  \textit{Dispense su gestione di progetto}:\\
    \href{https://www.math.unipd.it/~tullio/IS-1/2023/Dispense/T4.pdf}{https://www.math.unipd.it/~tullio/IS-1/2023/Dispense/T4.pdf}\\
    (Ultimo accesso: $2023-11-13$).
\end{itemize}

\section{Calendario di massima del progetto}
\subsection{Introduzione}
\paragraph{a  differenza di quanto visto nel preventivo costi,  il team cercherà di presentare la candidatura rtb nel periodo bla bla}
(preventivo  della lettera di candidatura e stima completamento lavori sempre della candidatura)


%\section{Stima dei costi di realizzazione}

\section{Rischi e loro mitigazione}
\paragraph{}Questa sezione di occupa di analizzare le difficoltà che si possono verificare durante lo svolgimento del progetto e che possono avere influenze sulla pianificazione delle attività, portando a rallentamenti e ostacoli nell'avanzamento.\\
Per poter individuare e gestire questi rischi, vengono di seguito esaminati e corredati da descrizione, previsione della loro occorrenza e grado di pericolosità e infine da misure di mitigazione degli effetti negativi nel caso si verifichino.
I rischi sono indicati con la notazione seguente: \\ R[Tipo]-[Numero] \ \  in cui:
\begin{itemize}
\item R sta per \textit{Rischio};
\item Tipo indica il tipo che può essere tra i seguenti:
    \begin{itemize}
	\item P (Personale);
	\item O (Organizzativo);
	\item T (Tecnologico).
    \end{itemize}
\item Numero indica il numero in quella categoria.
\end{itemize}


%PERSONALI

\subsection{Rischi personali}

% RISCHIO IMPEGNI PERSONALI

\subsubsection{Impegni e problemi personali}
{\renewcommand{\arraystretch}{1.5}
\begin{table}[h]
\begin{tabularx}{\textwidth}{c|X}
\textbf{ID Rischio} & RP-1 \\
\hline
\textbf{Rischio} & Impegni e problemi personali\\
\hline
\textbf{Descrizione} & Ogni componente del team ha impegni esterni e può avere problemi strettamente personali. Questo indica che qualche membro può non essere disponibile in certi momenti.\\
\hline
\textbf{Occorrenza} & Media\\
\hline
\textbf{Impatto} & Alto\\
\hline
\textbf{Misure di mitigazione} & I membri interessati si impegnano ad avvisare tempestivamente il gruppo; per far fronte a tale rischio si coprirà l’intervallo non produttivo del componente con una suddivisione omogenea tra i restanti colleghi delle attività rimaste in sospeso.
Riuscire a non spostare la \textit{milestone} è prioritario.\\
\end{tabularx}
\caption{Tabella rischi imprevisti e impegni personali}
\end{table}}




\subsubsection{Problemi fra i membri del gruppo}

{\renewcommand{\arraystretch}{1.5}
\begin{table}[h]
\begin{tabularx}{\textwidth}{c|X}
\textbf{ID Rischio} & RP-2 \\
\hline
\textbf{Rischio} & Problemi fra i membri del gruppo  \\
\hline
\textbf{Descrizione} & Possono verificarsi divergenze di pensiero tra i componenti del gruppo che rischiano di portare a discussioni. \\
\hline
\textbf{Occorrenza} & Media\\
\hline
\textbf{Impatto} & Medio \\
\hline
\textbf{Misure di mitigazione} & Le parti prese in causa esporranno il loro punto di vista in maniera educata; il Responsabile è tenuto a fare da moderatore. \\
\end{tabularx}
\caption{Tabella divergenze interne}
\end{table}
}


%ORGANIZZATIVI

\subsection{Rischi organizzativi}

\subsubsection{Sottostima del tempo necessario per una attività}

{\renewcommand{\arraystretch}{1.5}
\begin{table}[h]
\begin{tabularx}{\textwidth}{c|X}
\textbf{ID Rischio} & RO-1 \\
\hline
\textbf{Rischio} & Sottostima del tempo necessario per una attività\\
\hline
\textbf{Descrizione} & Il team può andare in contro ad una sottostima del tempo necessario per il completamento di un requisito o di un' attività.\\
\hline
\textbf{Occorrenza} & Alta\\
\hline
\textbf{Impatto} & Alto\\
\hline
\textbf{Misure di mitigazione} & Tale errore deve essere reso noto al team tempestivamente; chi ha disponibilità viene incaricato di fornire assistenza per minimizzare il ritardo nel completamento dell'obiettivo.\\
\end{tabularx}
\caption{Tabella sottostima del tempo}
\end{table}

\subsubsection{Stima errata dei costi}

{\renewcommand{\arraystretch}{1.5}
\begin{table}[H]
\begin{tabularx}{\textwidth}{c|X}
\textbf{ID Rischio} & RO-2 \\
\hline
\textbf{Rischio} & Stima errata dei costi \\
\hline
\textbf{Descrizione} & Potrebbe verificarsi una valutazione scorretta dei costi di incarico.\\
\hline
\textbf{Occorrenza} & Media\\
\hline
\textbf{Impatto} & Medio\\
\hline
\textbf{Misure di mitigazione} & Si provvederà a far uso dei diagrammi di Gantt per l'organizzazione delle attività lasciando dello "slack" tra le attività con dipendenze.\\
\end{tabularx}
\caption{Tabella stima errata dei costi}
\end{table}



\subsubsection{Disponibilità di lavoro non sfruttato}

{\renewcommand{\arraystretch}{1.5}
\begin{table}[H]
\begin{tabularx}{\textwidth}{c|X}
\textbf{ID Rischio} & RO-3 \\
\hline
\textbf{Rischio} & Disponibilità di lavoro non sfruttato \\
\hline
\textbf{Descrizione} & Caso in cui un membro del team si ritrova del tempo produttivo "libero" senza attività assegnate al proprio ruolo che lo impegnino.\\
\hline
\textbf{Occorrenza} & Media\\
\hline
\textbf{Impatto} & Medio\\
\hline
\textbf{Misure di mitigazione} & Il membro del team in questione viene invitato a farsi avanti e ad aiutare i membri in difficoltà o a farsi carico di attività non ancora assegnate anche non necessariamente legate al proprio ruolo in quel determinato \textit{sprint}. \\

\end{tabularx}
\caption{Tabella disponibilità non sfruttata}
\end{table}



%TECNOLOGICI

\subsection{Rischi tecnologici}

\subsubsection{Scarsa esperienza con le tecnologie del progetto}

{\renewcommand{\arraystretch}{1.5}
\begin{table}[H]
\begin{tabularx}{\textwidth}{c|X}
\textbf{ID Rischio} & RT-1 \\
\hline
\textbf{Rischio} & Scarsa esperienza con le tecnologie del progetto \\
\hline
\textbf{Descrizione} & Si possono verificare difficoltà di utilizzo di strumenti di lavoro o inesperienze\\
\hline
\textbf{Occorrenza} & Alta\\
\hline
\textbf{Impatto} & Alta\\
\hline
\textbf{Misure di mitigazione} & Il rischio non può essere evitato, dato che una buona esperienza con le tecnologie si ottiene con gli anni (tempo non a disposizione per questo progetto comunque complesso). Il primo passo è la comunicazione interna delle difficoltà più gravi.
\begin{itemize}
    \item Se le lacune riguardano le tecnologie usate abitualmente da AzzurroDigitale, è possibile richiedere un colloquio di formazione o supporto;
    \item Se le tecnologie sono note solo ad alcuni membri del gruppo, questi si impegnano a realizzare dei workshop per portare a regime le conoscenze degli altri componenti;
    \item Nel caso si parli di una tecnologia sconosciuta a tutti i membri, si prosegue cercando e studiando la guida utente e la documentazione associata.
\end{itemize}
\end{tabularx}
\caption{Tabella scarsa esperienza tecnologica}
\end{table}


\subsubsection{Guasti hardware e problematiche software}

{\renewcommand{\arraystretch}{1.5}
\begin{table}[H]
\begin{tabularx}{\textwidth}{c|X}
\textbf{ID Rischio} & RT-2 \\
\hline
\textbf{Rischio} & Guasti hardware e problematiche software \\
\hline
\textbf{Descrizione} & Si potrebbero presentare difficoltà legate all’hardware o al software utilizzati per cui un membro del gruppo risulta ostacolato nel portare a compimento un obiettivo o una attività. \\
\hline
\textbf{Occorrenza} & Bassa \\
\hline
\textbf{Impatto} & Medio\\
\hline
\textbf{Misure di mitigazione} & Comunicare tempestivamente il guasto al gruppo e se necessario chiedere aiuto e condivisione di un mezzo funzionante.\\
\end{tabularx}
\caption{Tabella guasti hardware e software}
\end{table}


\subsection{Valutazione efficacia delle misure}
\paragraph{}Questa sezione riassume la sezione di analisi dei rischi e valuta l'efficacia delle pratiche di mitigazioni degli stessi nel caso questi si siano verificati.\\
{\renewcommand{\arraystretch}{1.5}
\begin{table}[H]
\begin{tabularx}{\textwidth}{c|c|c|X}
\textbf{ID Rischio} & \textbf{Occorrenza effettiva} & \textbf{Impatto effettivo} & \textbf{\quantities{Efficacia misure di \\mitigazione}} \\
\hline
RP-1 & - & - & -\\
\hline
RP-2 & - & - & - \\
\hline
RO-1 & - & - & -\\
\hline
RO-2 & - & - & -\\
\hline
RO-3 & - & - & -\\
\hline
RT-1 & - & - & -\\
\hline
RT-2 & - & - & -\\


\end{tabularx}
\caption{Tabella riassuntiva delle misure di mitigazione}
\end{table}


\section{Pianificazione e modello di sviluppo}

\paragraph{}Il team decide di adottare il modello di sviluppo \textit{Agile} 
promuovendo un approccio incrementale al lavoro; questo consentirà di suddividere il progetto in compiti più gestibili, ovvero \textit{sprint}, ciascuno dei quali produrrà risultati funzionanti. Questo consente al team di progetto di reagire ai cambiamenti in modo più efficace.\\
\paragraph{}I periodi individuati sono \textit{sprint} di 14 giorni con seguente rotazione dei ruoli assunti dai componenti del gruppo.\\
Questo tipo di pianificazione permette di avere un orizzonte temporale limitato; ciò aiuta nel caso in cui vengano riscontrate difficoltà che causano danni ad attività che dipendono le une dalle altre, è possibile contenere questi ultimi e riorganizzare le attività nel periodo successivo.
\subsection{Requirements and Technology Baseline}
\paragraph{}Questo periodo antecedente la revisione RTB di progetto si focalizza sulla preparazione dei documenti (Analisi dei requisiti, Piano di progetto, Piano di qualifica, ampliamento Norme di progetto) e alla creazione del Proof of Concept per il progetto.

\subsubsection{Periodo da 2023-11-07 a 2023-11-20}



\subsubsection{Periodo da 2023-11-21 a 2023-12-04}


\subsubsection{Periodo da 2023-12-05 a 2023-12-18}





\subsection{Product Baseline}



\subsubsection{Periodo da 2023-12-19 a 2024-01-01}



\subsubsection{Periodo da 2024-01-02 a 2024-01-15}



\subsubsection{Periodo da 2024-01-16 a 2024-01-29}



\subsubsection{Periodo da 2024-01-30 a 2024-02-12}



\subsubsection{Periodo da 2024-02-13 a 2024-02-26}



\subsubsection{Periodo da 2024-02-27 a 2024-03-11}


\subsubsection{Periodo da 2024-03-12 a 2024-03-25}



\subsubsection{Periodo da 2024-03-26 a 2024-04-08}



\section{Preventivo}

%Per queste voci si rimanda al documento “Preventivo costi e assunzione impegni”, presente nella repository Presentazio




\section{Consuntivo}

\section{Retrospettiva generale}
\subsection{Gestione delle risorse}
\subsubsection{Tempo}
\subsubsection{Budget}
\subsection{Aspetti positivi}
\subsection{Aspetti negativi}
\color{gray}\paragraph{}pratica del vincolo di un ruolo, può non far sfruttare tutta la disponibilità di lavoro dei componenti, etc.
\color{black}
\subsection{Conclusioni}

\end{document}